\documentclass[a4paper,11pt]{article}

%%%%%%%%%%%%%%%%%%%%%%%%%%%%%%%%%%%%%%%%%%%%%%%%%%%%%%%%%%%%%%%%%%%%%%%%
% Paquetes utilizados
%%%%%%%%%%%%%%%%%%%%%%%%%%%%%%%%%%%%%%%%%%%%%%%%%%%%%%%%%%%%%%%%%%%%%%%%
% Soporte para el lenguaje español
\usepackage{textcomp}
\usepackage[utf8]{inputenc}
\usepackage[T1]{fontenc}
\usepackage[spanish]{babel}

% Configuración de página
\usepackage[lmargin=1in, rmargin=1in]{geometry}
\setlength{\parskip}{0.75em}
\setlength{\parindent}{1em}

% Graficos
\usepackage{graphicx}

% Encabezados y pies de pagina
\usepackage{fancyhdr}
\setlength{\headheight}{15.2pt}
\pagestyle{fancy}
\fancyhf{}
\lhead{75.67 - Sist. Autom. de Diag. y Detección de Fallas I}
\rhead{Trabajo Práctico Final}
\cfoot{\thepage}

% Tablas
\usepackage{longtable}
\usepackage{array}
\newcolumntype{L}[1]{>{\raggedright\let\newline\\\arraybackslash\hspace{0pt}}m{#1}}

% Urls
\usepackage[hidelinks]{hyperref}

% Referencias
\usepackage[nottoc,numbib]{tocbibind}

\begin{document}

%%%%%%%%%%%%%%%%%%%%%%%%%%%%%%%%%%%%%%%%%%%%%%%%%%%%%%%%%%%%%%%%%%%%%%%%
% Titulo
%%%%%%%%%%%%%%%%%%%%%%%%%%%%%%%%%%%%%%%%%%%%%%%%%%%%%%%%%%%%%%%%%%%%%%%%

\thispagestyle{empty}

\begin{titlepage}

  \newcommand{\HRule}{\rule{\linewidth}{0.5mm}}
  \newenvironment{bottompar}{\par\vspace*{\fill}}{\clearpage}

  \center

  \textsc{\LARGE Universidad de Buenos Aires}\\[1.0cm]
  \textsc{\Large Facultad de Ingeniería}\\[1.5cm]


  \textsc{\large 75.67 - Sistemas Automáticos de Diagnóstico y Detección de Fallas I}\\[0.25cm]
  \HRule\\[0.25cm]
  {\huge \bfseries Trabajo Práctico Final}\\
  \HRule\\
  Sistema experto sobre pesca deportiva en ríos y lagunas

  \begin{bottompar}
    \begin{minipage}[t]{.45\linewidth}
      \begin{flushleft}
        {\bfseries Autores:}

        Arana, Andrés          - P. 86.203

        Laghi, Guido           - P. 82.449

        Pérez, Sebastián       - P. 85.191
      \end{flushleft}
    \end{minipage}
    \hfill
    \begin{minipage}[t]{.45\linewidth}
      \begin{flushright}
        {\bfseries Profesor:}

        Merlino, Hernán Daniel
      \end{flushright}
    \end{minipage}
  \end{bottompar}

\end{titlepage}

%%%%%%%%%%%%%%%%%%%%%%%%%%%%%%%%%%%%%%%%%%%%%%%%%%%%%%%%%%%%%%%%%%%%%%%%
% Documento
%%%%%%%%%%%%%%%%%%%%%%%%%%%%%%%%%%%%%%%%%%%%%%%%%%%%%%%%%%%%%%%%%%%%%%%%

%-----------------------------------------------------------------------
% Abstract
%-----------------------------------------------------------------------
\begin{abstract}

  El presente documento detalla el desarrollo del trabajo práctico final para
  la materia 75.67 - Sistemas Automáticos de Diagnóstico y Detección de Fallas
  I de la carrera de Ingeniería en Informática de la Universidad de Buenos
  Aires. El mismo consiste en la elicitación, formalización, modelado e
  implementación de un sistema experto sobre la pesca deportiva de agua dulce
  en la República Argentina, utilizando la metodología IDEAL analizada en la
  materia.

\end{abstract}
\clearpage

%-----------------------------------------------------------------------
% Tabla de contenidos
%-----------------------------------------------------------------------
\tableofcontents
\clearpage

%-----------------------------------------------------------------------
% Desarrollo
%-----------------------------------------------------------------------
\section{Viabilidad}

\subsection{Objetivo y alcance}

El objetivo del proyecto es la elaboración de un sistema experto que modele el
conocimiento sobre la pesca deportiva de especies de agua dulce en la República
Argentina, permitiendo realizar consultas para determinar el equipamiento,
lugar, tiempo y especie apropiados en base a las preferencias del usuario.

El alcance del proyecto es la deducción, conceptualización y formalización del
conocimiento, así como también la implementación del sistema de un sistema
básico por consola que permita realizar consultas sobre la base de
conocimientos. Se permitirá definir diferentes características de una salida de
pesca, y el sistema recomendará las mejores opciones de equipo, especies,
fechas y horas en función de la selección interactiva del usuario.

\subsection{Experto}

El experto a consultar en el desarrollo del sistema es Clodomiro Morinigo,
pescador en modalidad deportiva con más de 50 años de experiencia en la pesca
de especies nativas de Buenos Aires y el litoral argentino.

Se coordinó con el experto la posibilidad de realizar, si fuese necesario,
reuniones semanales los días sábados, domingos y lunes de hasta 3 horas de
duración, a partir de las 19hs. Se garantizó la disponibilidad para realizar
hasta una reunión por semana.

\subsection{Requerimientos}

Se desarrollará una aplicación interactiva de consola que permita definir
características de una salida de pesca.En base a las características definidas,
se podrán consultar las mejores opciones de equipo, especies, fecha y horas
para las mismas. Las características de la salida se refinarán de manera
interactiva, pudiendo confirmar y retractar atributos de la salida y obteniendo
las opciones actualizadas en tiempo real.

\section{Desarrollo}

\subsection{Conceptualización}\label{sec:concept}

Se decidió comenzar el proceso de deducción y conceptualización con una
entrevista abierta con el experto. En la misma se extrajeron los conceptos
principales y algunos atributos que se detallan en la sección~\ref{sec:tcav}.

Una vez elaborada la tabla de conceptos, atributos y valores inicial, se
realizó una investigación sobre el cuerpo de conocimiento público. En
particular, se utilizaron dos sitios recomendados por el experto:
\url{http://www.irapescar.com/} y \url{http://www.pescaargentina.com.ar/}.

El objetivo de esta investigación fue conseguir una base de conocimiento previo
para poder realizar un diccionario inicial de conceptos y atributos, además de
completar la tabla de conceptos, atributos y valores con mayor detalle. Se puso
especial énfasis en que esta fase fuese exploratoria: se recopilaron conceptos,
atributos y valores sin realizar ningún tipo de filtro. Posteriormente, algunos
de estos fueron eliminados del modelo en base a la entrevista subsiguiente con
el experto.

Una vez terminada la fase exploratoria del conocimiento público, se realizó una
nueva entrevista con el experto, con formato cerrado, para validar la tabla de
la sección~\ref{sec:tcav} y el diccionario de la sección~\ref{sec:dca}. En esta
entrevista se eliminaron conceptos, atributos y valores irrelevantes, se
redefinieron algunos atributos y finalmente se aprobaron los contenidos de
ambas secciones como se detallan en el presente documento.

En la misma entrevista se realizó un emparrillado de manera interactiva. Se
transcribe el mismo en la sección~\ref{sec:emp}.

En base a todo el conocimiento formalizado hasta el momento, se organizó una
última entrevista para aprobar los modelos desarrollados

\subsection{Formalización}

En esta sección se detallan todos los modelos realizados en el desarrollo del
trabajo, en su versión final aprobada por el experto. Para una visión
cronológica de los borradores presentados y su posterior aprobación, ver la
sección~\ref{sec:concept}.

\subsubsection{Tablas de conceptos, atributos y valores}\label{sec:tcav}

\begin{table}[h!]
\centering
\begin{tabular}{ | p{0.15\linewidth} | p{0.2\linewidth} | p{0.65\linewidth} | }
  \hline
  \textbf{Concepto} & \textbf{Atributo} & \textbf{Valores} \\ \hline
  Pez & Especie & Pejerrey, Boga, Surubí, Tararira, Dorado, Sábalo, Carpa, Tarucha Azul \\\hline
  Pez & Estación & Primavera, Verano, Otoño, Invierno, Todas \\\hline
  Pez & Temperatura & Templado, Cálido \\\hline
  Pez & Estado del agua & Revuelta, Normal, Tranquila, Corredera, Pozo \\\hline
  Pez & Horario ideal & Madrugada, Mañana, Mediodía, Tarde, Noche, Todos \\\hline
  Pez & Tamaño & Chico, Mediano, Grande \\\hline
\end{tabular}
\end{table}

\begin{table}[h!]
\centering
\begin{tabular}{ | p{0.15\linewidth} | p{0.2\linewidth} | p{0.65\linewidth} | }
  \hline
  \textbf{Concepto} & \textbf{Atributo} & \textbf{Valores} \\ \hline
  Lancha & Propulsión & Motor, Remo \\\hline
  Lancha & Tamaño & Entre 1 y 7 metros \\\hline
\end{tabular}
\end{table}

\begin{table}[h!]
\centering
\begin{tabular}{ | p{0.15\linewidth} | p{0.2\linewidth} | p{0.65\linewidth} | }
  \hline
  \textbf{Concepto} & \textbf{Atributo} & \textbf{Valores} \\\hline
  Caña & Rigidez & Flexible, Rígida \\\hline
  Caña & Línea & Entre 0.2 y 1mm \\\hline
  Caña & Carnada & Masa, Lombriz, Mojarra, Bagre, Anguila, Señuelo \\\hline
  Caña & Telescópica & Si, No \\\hline
  Caña & Tamaño & 1,5m, 2,5m, 4m \\\hline
  Caña & Plomada & Pequeña, Mediana, Grande \\\hline
  Caña & Anzuelo & N°3, N°4, N°5, N°6, N°7, N°8, N°9, N°10 \\\hline
\end{tabular}
\end{table}

\begin{table}[h!]
\centering
\begin{tabular}{ | p{0.15\linewidth} | p{0.2\linewidth} | p{0.65\linewidth} | }
  \hline
  \textbf{Concepto} & \textbf{Atributo} & \textbf{Valores} \\\hline
  Reel & Tipo & Frontal, Malacate \\\hline
  Reel & Tamaño & Chico, Mediano, Grande \\\hline
  Reel & Freno & Si, No \\\hline
\end{tabular}
\end{table}

\begin{table}[h!]
\centering
\begin{tabular}{ | p{0.15\linewidth} | p{0.2\linewidth} | p{0.65\linewidth} | }
  \hline
  \textbf{Concepto} & \textbf{Atributo} & \textbf{Valores} \\\hline
  Ubicación & Nombre & \\\hline
  Ubicación & Tipo & Río, Lago, Laguna \\\hline
  Ubicación & Profundidad máx. & \\\hline
  Ubicación & Coordenadas & \\\hline
\end{tabular}
\end{table}

\subsubsection{Diccionario de conceptos y atributos}\label{sec:dca}

\begin{description}

  \item [Pez] Animales vertebrados acuáticos. En el contexto de este trabajo
    nos centraremos en aquellos peces que son pescados en la República
    Argentina.

  \item [Especie] Unidad básica de clasificación biológica de los peces. En el
    contexto de este trabajo nos centraremos en aquellas especies de peces que
    son pescados en la República Argentina.

  \item[Estación] Estación del año óptima en la que se puede pescar con mayor
    probabilidad una determinada especie.

  \item[Temperatura] Temperatura óptima del agua en la que se puede pescar con
    mayor probabilidad una determinada especie. En la zona del litoral de la
    República Argentina sólo se encuentran temperaturas templadas y cálidas.

  \item[Estado del agua] Ciertas especies, debido a sus hábitos alimenticios,
    se encuentran con mayor probabilidad en zonas de un cuerpo de agua en donde
    se forman ciertos fenómenos de flujo hídrico. Los valores posibles son
    \emph{Revuelta} cuando el flujo es turbulento, \emph{Normal} cuando el
    flujo es estable, \emph{Tranquila} cuando no hay flujo, \emph{Corredera}
    cuando el flujo es elevado y monodireccional, \emph{Pozo} cuando la
    profundidad del flujo es mucho mayor que en la cercanía.

  \item[Horario ideal] Horario del día ideal para capturar una especie
    particular, discretizado a \emph{Mañana} entre las 7:00 y las 12:00,
    \emph{Mediodía} entre 12:00 y las 15:00, \emph{Tarde} entre las 15:00 y las
    19:00, \emph{Noche} entre las 19:00 y las 12:00 y \emph{Madrugada} entre
    las 12:00 y las 7:00.

  \item[Tamaño] Peso promedio considerando una distribución normal de una
    cierta especie, discretizado a \emph{Chico} para pesos menores a 4 kg,
    \emph{Mediano} para pesos entre 4kg y 7kg y \emph{Grande} para pesos
    mayores a 7kg.

  \item[Lancha] Embarcación utilizada para pescar.

  \item[Propulsión] Tipo de propulsión utilizada para mover la lancha. Cada
    método de propulsión tiene sus propias características \emph{Motor} es más
    cómodo y rápido, pero debe apagarse para poder realizar algunos tipos de
    pesca. \emph{Remo} es más lento, pero puede utilizarse en cualquier
    momento.

  \item[Caña] Aparejo utilizado para pescar.

  \item[Rigidez] Flexibilidad de la caña. Especies más grandes y más pesadas
    requieren una mayor flexibilidad para evitar quiebres y fallas en el
    equipo.

  \item[Línea] Hilo de nylon utilizado en la caña para unir al anzuelo. En
    particular, nos interesa el diámetro seccional del hilo, entre 0.2mm y 1mm.
    Mayores diámetros son utilizados para especies más grandes y pesadas,
    aunque también hay que tener en cuenta la violencia con la que se comporta
    la especie cuando es encañada.

  \item[Carnada] Tipo de carnada que una caña lleva. Puede ser una masa
    proteica, lombriz o algún tipo de carnada viva.

  \item[Telescópica] Una caña telescópica permite la extensión dinámica de la
    misma para realizar tiros más profundos y lejanos.

  \item[Tamaño] Longitud de la caña. Cañas más largas permiten una mayor
    llegada, pero pueden romperse más fácilmente con especies más violentas o
    grandes.

  \item[Plomada] Peso adicional que se ubica en la cercanía del anzuelo para
    mantener la carnada o el señuelo a una profundidad deseada.

  \item[Anzuelo] Tamaño del anzuelo utilizado. Anzuelos más pequeños son
    utilizados para especies pequeñas y viceversa.

  \item[Reel] Dispositivo que une la línea a la caña, y controla la entrega y
    toma de línea.

  \item[Tipo] Tipo de reel. Los reeles frontales son más livianos y fáciles de
    usar, pero algo más vulnerables a fallas cuando la especie pescada es más
    grande o violenta.

  \item[Tamaño] Tamaño del reel. Reeles más grandes son utilizados para
    especies más grandes, tienen más resistencia pero menor sensibilidad.

  \item[Freno] Dispositivo que retrasa la entrega de línea, de manera que el
    pez debe realizar más fuerza para tomar línea.

\end{description}

\subsubsection{Emparrillado}\label{sec:emp}

\begin{table}[h!]
\centering
\begin{tabular}{ | p{0.15\linewidth} | p{0.10\linewidth} | p{0.10\linewidth} | p{0.10\linewidth} | p{0.10\linewidth} | p{0.10\linewidth} | p{0.10\linewidth} | p{0.10\linewidth} | }
  \hline
  & \textbf{Pejerrey} & \textbf{Boga} & \textbf{Surubí} & \textbf{Tararira} & \textbf{Dorado} & \textbf{Carpa} & \textbf{Tarucha Azul} \\\hline
  \textbf{Carnada} & Mojarra & Masa & Anguila & Mojarra & Bagre & Lombriz & Mojarra \\\hline
  \textbf{Línea} & 0.2 & 0.2 & 0.4 & 0.3 & 0.4 & 0.2 & 0.4 \\\hline
  \textbf{Señuelo} & No & No & Si & Si & Si & No & Si \\\hline
  \textbf{Plomada} & No & Si & Si & Si & Si & Si & Si \\\hline
  \textbf{Reel} & Chico & Mediano & Grande & Mediano & Mediano & Chico & Mediano \\\hline
  \textbf{Caña} & 4m & 2.5m & 1.5m & 1.5m & 1.5m & 2.5m & 1.5m \\\hline
  \textbf{Telescópica} & Si & No & No & No & No & No & No \\\hline
  \textbf{Anzuelo} & 3 & 7 & 10 & 7 & 7 & 3 & 8 \\\hline
  \textbf{Estación} & Invierno & Verano & Verano & Verano & Todas & Verano & Verano \\\hline
  \textbf{Agua} & Revuelta & Normal & Normal & Tranquila & Corredera & Normal & Tranquila \\\hline
  \textbf{Temperatura} & Templado & Cálido & Cálido & Cálido & Cálido & Cálido & Cálido \\\hline
  \textbf{Horario} & Mañana & Todo & Madrugada, Noche & Noche & Mañana, Tarde & Todo & Noche \\\hline
  \textbf{Profundidad} & Superficial & Profundo & Profundo & Media & Media & Todo & Media \\\hline
  \textbf{Ubicación} & Laguna & Río & Río & Laguna & Río & Laguna & Río \\\hline
  \textbf{Tamaño} & Chico & Mediano & Grande & Mediano & Mediano & Chico & Mediano \\\hline
  \textbf{Freno} & No & Si & Si & Si & Si & Si & Si \\\hline
\end{tabular}
\end{table}

\section{Conclusiones}

Durante la fase de deducción de conocimiento pudimos comprobar la dificultad de
externalizar el conocimiento interno del experto.

En primer lugar, luego de la primer entrevista, tuvimos que volcarnos al
repositorio de conocimiento público para hacer definiciones certeras de cada
concepto, atributo y valor: el experto no podía externalizar ese conocimiento
debido a que su perspectiva es la de alguien que intuitivamente conoce dichos
elementos, por lo que le era difícil explicar a alguien que no tiene el
conocimiento qué significa cada término.

Por otro lado, descubrimos muchas reglas que el experto conocía intuitivamente,
y que ni siquiera era consciente que aplicaba. Por ejemplo, inicialmente
relevamos que el tamaño del reel dependía de la especie que se quería pescar.
Sin embargo, luego de examinar el emparrillado, comprendimos que el tamaño del
reel depende en realidad mayoritariamente del tamaño promedio de la especie,
así como también de la violencia con la que se comporta la misma en el
encañado. Cuando validamos la regla con el experto, le pareció una apreciación
obvia.

\end{document}
